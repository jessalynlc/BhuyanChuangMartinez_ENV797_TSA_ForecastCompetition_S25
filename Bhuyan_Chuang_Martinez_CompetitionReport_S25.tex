% Options for packages loaded elsewhere
\PassOptionsToPackage{unicode}{hyperref}
\PassOptionsToPackage{hyphens}{url}
%
\documentclass[
]{article}
\usepackage{amsmath,amssymb}
\usepackage{iftex}
\ifPDFTeX
  \usepackage[T1]{fontenc}
  \usepackage[utf8]{inputenc}
  \usepackage{textcomp} % provide euro and other symbols
\else % if luatex or xetex
  \usepackage{unicode-math} % this also loads fontspec
  \defaultfontfeatures{Scale=MatchLowercase}
  \defaultfontfeatures[\rmfamily]{Ligatures=TeX,Scale=1}
\fi
\usepackage{lmodern}
\ifPDFTeX\else
  % xetex/luatex font selection
\fi
% Use upquote if available, for straight quotes in verbatim environments
\IfFileExists{upquote.sty}{\usepackage{upquote}}{}
\IfFileExists{microtype.sty}{% use microtype if available
  \usepackage[]{microtype}
  \UseMicrotypeSet[protrusion]{basicmath} % disable protrusion for tt fonts
}{}
\makeatletter
\@ifundefined{KOMAClassName}{% if non-KOMA class
  \IfFileExists{parskip.sty}{%
    \usepackage{parskip}
  }{% else
    \setlength{\parindent}{0pt}
    \setlength{\parskip}{6pt plus 2pt minus 1pt}}
}{% if KOMA class
  \KOMAoptions{parskip=half}}
\makeatother
\usepackage{xcolor}
\usepackage[margin=1in]{geometry}
\usepackage{color}
\usepackage{fancyvrb}
\newcommand{\VerbBar}{|}
\newcommand{\VERB}{\Verb[commandchars=\\\{\}]}
\DefineVerbatimEnvironment{Highlighting}{Verbatim}{commandchars=\\\{\}}
% Add ',fontsize=\small' for more characters per line
\usepackage{framed}
\definecolor{shadecolor}{RGB}{248,248,248}
\newenvironment{Shaded}{\begin{snugshade}}{\end{snugshade}}
\newcommand{\AlertTok}[1]{\textcolor[rgb]{0.94,0.16,0.16}{#1}}
\newcommand{\AnnotationTok}[1]{\textcolor[rgb]{0.56,0.35,0.01}{\textbf{\textit{#1}}}}
\newcommand{\AttributeTok}[1]{\textcolor[rgb]{0.13,0.29,0.53}{#1}}
\newcommand{\BaseNTok}[1]{\textcolor[rgb]{0.00,0.00,0.81}{#1}}
\newcommand{\BuiltInTok}[1]{#1}
\newcommand{\CharTok}[1]{\textcolor[rgb]{0.31,0.60,0.02}{#1}}
\newcommand{\CommentTok}[1]{\textcolor[rgb]{0.56,0.35,0.01}{\textit{#1}}}
\newcommand{\CommentVarTok}[1]{\textcolor[rgb]{0.56,0.35,0.01}{\textbf{\textit{#1}}}}
\newcommand{\ConstantTok}[1]{\textcolor[rgb]{0.56,0.35,0.01}{#1}}
\newcommand{\ControlFlowTok}[1]{\textcolor[rgb]{0.13,0.29,0.53}{\textbf{#1}}}
\newcommand{\DataTypeTok}[1]{\textcolor[rgb]{0.13,0.29,0.53}{#1}}
\newcommand{\DecValTok}[1]{\textcolor[rgb]{0.00,0.00,0.81}{#1}}
\newcommand{\DocumentationTok}[1]{\textcolor[rgb]{0.56,0.35,0.01}{\textbf{\textit{#1}}}}
\newcommand{\ErrorTok}[1]{\textcolor[rgb]{0.64,0.00,0.00}{\textbf{#1}}}
\newcommand{\ExtensionTok}[1]{#1}
\newcommand{\FloatTok}[1]{\textcolor[rgb]{0.00,0.00,0.81}{#1}}
\newcommand{\FunctionTok}[1]{\textcolor[rgb]{0.13,0.29,0.53}{\textbf{#1}}}
\newcommand{\ImportTok}[1]{#1}
\newcommand{\InformationTok}[1]{\textcolor[rgb]{0.56,0.35,0.01}{\textbf{\textit{#1}}}}
\newcommand{\KeywordTok}[1]{\textcolor[rgb]{0.13,0.29,0.53}{\textbf{#1}}}
\newcommand{\NormalTok}[1]{#1}
\newcommand{\OperatorTok}[1]{\textcolor[rgb]{0.81,0.36,0.00}{\textbf{#1}}}
\newcommand{\OtherTok}[1]{\textcolor[rgb]{0.56,0.35,0.01}{#1}}
\newcommand{\PreprocessorTok}[1]{\textcolor[rgb]{0.56,0.35,0.01}{\textit{#1}}}
\newcommand{\RegionMarkerTok}[1]{#1}
\newcommand{\SpecialCharTok}[1]{\textcolor[rgb]{0.81,0.36,0.00}{\textbf{#1}}}
\newcommand{\SpecialStringTok}[1]{\textcolor[rgb]{0.31,0.60,0.02}{#1}}
\newcommand{\StringTok}[1]{\textcolor[rgb]{0.31,0.60,0.02}{#1}}
\newcommand{\VariableTok}[1]{\textcolor[rgb]{0.00,0.00,0.00}{#1}}
\newcommand{\VerbatimStringTok}[1]{\textcolor[rgb]{0.31,0.60,0.02}{#1}}
\newcommand{\WarningTok}[1]{\textcolor[rgb]{0.56,0.35,0.01}{\textbf{\textit{#1}}}}
\usepackage{longtable,booktabs,array}
\usepackage{calc} % for calculating minipage widths
% Correct order of tables after \paragraph or \subparagraph
\usepackage{etoolbox}
\makeatletter
\patchcmd\longtable{\par}{\if@noskipsec\mbox{}\fi\par}{}{}
\makeatother
% Allow footnotes in longtable head/foot
\IfFileExists{footnotehyper.sty}{\usepackage{footnotehyper}}{\usepackage{footnote}}
\makesavenoteenv{longtable}
\usepackage{graphicx}
\makeatletter
\def\maxwidth{\ifdim\Gin@nat@width>\linewidth\linewidth\else\Gin@nat@width\fi}
\def\maxheight{\ifdim\Gin@nat@height>\textheight\textheight\else\Gin@nat@height\fi}
\makeatother
% Scale images if necessary, so that they will not overflow the page
% margins by default, and it is still possible to overwrite the defaults
% using explicit options in \includegraphics[width, height, ...]{}
\setkeys{Gin}{width=\maxwidth,height=\maxheight,keepaspectratio}
% Set default figure placement to htbp
\makeatletter
\def\fps@figure{htbp}
\makeatother
\setlength{\emergencystretch}{3em} % prevent overfull lines
\providecommand{\tightlist}{%
  \setlength{\itemsep}{0pt}\setlength{\parskip}{0pt}}
\setcounter{secnumdepth}{-\maxdimen} % remove section numbering
\usepackage{booktabs}
\usepackage{longtable}
\usepackage{array}
\usepackage{multirow}
\usepackage{wrapfig}
\usepackage{float}
\usepackage{colortbl}
\usepackage{pdflscape}
\usepackage{tabu}
\usepackage{threeparttable}
\usepackage{threeparttablex}
\usepackage[normalem]{ulem}
\usepackage{makecell}
\usepackage{xcolor}
\ifLuaTeX
  \usepackage{selnolig}  % disable illegal ligatures
\fi
\usepackage{bookmark}
\IfFileExists{xurl.sty}{\usepackage{xurl}}{} % add URL line breaks if available
\urlstyle{same}
\hypersetup{
  pdftitle={Bhuyan\_Chuang\_Martinez\_CompetitionReport\_S25},
  pdfauthor={Mazhar Bhuyan, Jessalyn Chuang, Sayra Martinez},
  hidelinks,
  pdfcreator={LaTeX via pandoc}}

\title{Bhuyan\_Chuang\_Martinez\_CompetitionReport\_S25}
\author{Mazhar Bhuyan, Jessalyn Chuang, Sayra Martinez}
\date{2025-04-25}

\begin{document}
\maketitle

\subsection{GitHub Repository}\label{github-repository}

\href{https://github.com/jessalynlc/BhuyanChuangMartinez_ENV797_TSA_ForecastCompetition_S25}{Project
Repository}
\n[Repository Fork](\url{https://github.com/MazharBhuyan/BhuyanChuangMartinez_ENV797_TSA_ForecastCompetition_S25})
\n*Note that work was also done in a fork of this project and we were
unable to merge with the main repository!

\subsection{Introduction}\label{introduction}

This report presents our forecasting approach for the ENV797 TSA
Forecasting Competition. Our objective was to forecast daily electricity
demand using time series models and outperform the benchmark STL+ETS
model. Model evaluation was based on minimizing the Mean Absolute
Percentage Error (MAPE) over a validation set.

\subsection{Data Description}\label{data-description}

The dataset included hourly electricity load, temperature, and relative
humidity from January 2005 to December 2010. After cleaning and removing
missing values, we:

\begin{itemize}
\tightlist
\item
  Aggregated the data to daily averages.
\item
  Created an \texttt{msts} time series object with weekly (7-day) and
  yearly (365.25-day) seasonality.
\item
  Used January 1, 2005 to December 31, 2009 for model training.
\item
  Used January 1, 2010 to February 28, 2010 as a validation set.
\end{itemize}

The dataset included hourly electricity load, temperature, and relative
humidity from January 2005 to December 2010. After cleaning and removing
missing values, we:

\begin{itemize}
\tightlist
\item
  Aggregated the data to daily averages.
\item
  Created an \texttt{msts} time series object with weekly (7-day) and
  yearly (365.25-day) seasonality.
\item
  Used January 1, 2005 to December 31, 2009 for model training.
\item
  Used January 1, 2010 to February 28, 2010 as a validation set.
\end{itemize}

\subsection{Top 5 Forecasting Models}\label{top-5-forecasting-models}

We tested several models in a systematic manner and evaluated them based
on validation MAPE.

\begin{enumerate}
\def\labelenumi{\arabic{enumi}.}
\tightlist
\item
  \textbf{Model 1: NNAR + Fourier (K=2,8)} \emph{(Selected Best Model)}
\item
  Model 2: NNAR + Fourier (K=2,12)
\item
  Model 3: NNAR + Fourier (K=2,12) + Temp
\item
  Model 4: NNAR + Fourier (K=3,18)
\item
  Model 5: TBATS
\end{enumerate}

We also experimented with an ARIMA + NNAR hybrid model, but it was not
among the top 5 based on MAPE.

\subsection{Modeling \& Forecast
Results}\label{modeling-forecast-results}

\subsubsection{Model 1: NNAR + Fourier (K =
2,8)}\label{model-1-nnar-fourier-k-28}

We tried this model after trying out more complex NNAR + Fourier models.
We chose to use NNAR + Fourier (K= 2,8) for forecasting load because it
provided a strong balance between model complexity and performance. The
Fourier terms capture load's seasonal patterns, such as daily and yearly
cycles, while the NNAR model handles the nonlinear and irregular
behaviors driven by weather/holidays/human activity. By selecting K = 2
and 8, we introduced enough flexibility to represent both broad and
finer seasonal trends without overfitting, allowing the neural network
to focus on learning residual variations while not over/underfitting to
the data.

\begin{Shaded}
\begin{Highlighting}[]
\NormalTok{K1 }\OtherTok{\textless{}{-}} \FunctionTok{c}\NormalTok{(}\DecValTok{2}\NormalTok{,}\DecValTok{8}\NormalTok{)}
\NormalTok{horizon }\OtherTok{\textless{}{-}} \FunctionTok{length}\NormalTok{(ts\_daily\_test)}
\NormalTok{NN\_fit\_k28 }\OtherTok{\textless{}{-}} \FunctionTok{nnetar}\NormalTok{(ts\_daily\_train, }\AttributeTok{p =} \DecValTok{2}\NormalTok{, }\AttributeTok{P =} \DecValTok{2}\NormalTok{, }\AttributeTok{xreg =} \FunctionTok{fourier}\NormalTok{(ts\_daily\_train, }\AttributeTok{K =}\NormalTok{ K1))}
\NormalTok{NN\_for\_k28 }\OtherTok{\textless{}{-}} \FunctionTok{forecast}\NormalTok{(NN\_fit\_k28, }\AttributeTok{h =}\NormalTok{ horizon, }\AttributeTok{xreg =} \FunctionTok{fourier}\NormalTok{(ts\_daily\_train, }\AttributeTok{K =}\NormalTok{ K1, }\AttributeTok{h =}\NormalTok{ horizon))}
\FunctionTok{autoplot}\NormalTok{(ts\_daily\_test) }\SpecialCharTok{+} \FunctionTok{autolayer}\NormalTok{(NN\_for\_k28, }\AttributeTok{series =} \StringTok{"Model 1 Forecast"}\NormalTok{)}
\end{Highlighting}
\end{Shaded}

\begin{center}\includegraphics{Bhuyan_Chuang_Martinez_CompetitionReport_S25_files/figure-latex/model1-nnar-k28-1} \end{center}

\begin{Shaded}
\begin{Highlighting}[]
\FunctionTok{accuracy}\NormalTok{(NN\_for\_k28, ts\_daily\_test)}
\end{Highlighting}
\end{Shaded}

\begin{verbatim}
##                      ME     RMSE      MAE        MPE      MAPE      MASE
## Training set  -1.175759  263.097  190.050 -0.9871232  5.653911 0.2470082
## Test set     810.755950 1967.805 1642.677  8.7475449 32.499205 2.1349881
##                     ACF1 Theil's U
## Training set -0.03411675        NA
## Test set      0.85904423  2.385696
\end{verbatim}

\subsubsection{Model 2: NNAR + Fourier (K =
2,12)}\label{model-2-nnar-fourier-k-212}

We first tested NNAR + Fourier (K=2,12) and considered it our baseline
model for forecasting load. This setup allowed us to capture the major
seasonal patterns in the data using Fourier terms while using the
flexibility of a neural network to model nonlinearities and irregular
fluctuations. Choosing K = 2 and 12 offered a broader range of seasonal
harmonics to start.

\begin{Shaded}
\begin{Highlighting}[]
\NormalTok{K2 }\OtherTok{\textless{}{-}} \FunctionTok{c}\NormalTok{(}\DecValTok{2}\NormalTok{,}\DecValTok{12}\NormalTok{)}
\NormalTok{NN\_fit\_k212 }\OtherTok{\textless{}{-}} \FunctionTok{nnetar}\NormalTok{(ts\_daily\_train, }\AttributeTok{p =} \DecValTok{2}\NormalTok{, }\AttributeTok{P =} \DecValTok{2}\NormalTok{, }\AttributeTok{xreg =} \FunctionTok{fourier}\NormalTok{(ts\_daily\_train, }\AttributeTok{K =}\NormalTok{ K2))}
\NormalTok{NN\_for\_k212 }\OtherTok{\textless{}{-}} \FunctionTok{forecast}\NormalTok{(NN\_fit\_k212, }\AttributeTok{h =}\NormalTok{ horizon, }\AttributeTok{xreg =} \FunctionTok{fourier}\NormalTok{(ts\_daily\_train, }\AttributeTok{K =}\NormalTok{ K2, }\AttributeTok{h =}\NormalTok{ horizon))}
\FunctionTok{autoplot}\NormalTok{(ts\_daily\_test) }\SpecialCharTok{+} \FunctionTok{autolayer}\NormalTok{(NN\_for\_k212, }\AttributeTok{series =} \StringTok{"Model 2 Forecast"}\NormalTok{)}
\end{Highlighting}
\end{Shaded}

\begin{center}\includegraphics{Bhuyan_Chuang_Martinez_CompetitionReport_S25_files/figure-latex/model2-nnar-k212-1} \end{center}

\begin{Shaded}
\begin{Highlighting}[]
\FunctionTok{accuracy}\NormalTok{(NN\_for\_k212, ts\_daily\_test)}
\end{Highlighting}
\end{Shaded}

\begin{verbatim}
##                       ME      RMSE       MAE        MPE      MAPE      MASE
## Training set   0.1853845  152.2358  109.0513 -0.4475076  3.298475 0.1417341
## Test set     775.2930241 2038.3059 1737.2961  7.9584180 34.338455 2.2579650
##                    ACF1 Theil's U
## Training set -0.1154700        NA
## Test set      0.8505419  2.455614
\end{verbatim}

\subsubsection{Model 3: NNAR + Fourier (K = 2,12) +
Temp}\label{model-3-nnar-fourier-k-212-temp}

Including temperature as a regressor can help capture demand variations
driven by weather. However, adding multiple regressors like temperature
and humidity risks introducing multicollinearity, especially when both
variables are correlated (e.g., hot days often being humid).
Multicollinearity can lead to unstable neural network training,
overfitting, and degraded forecasting performance if not carefully
regularized. This model showed that temperature alone did not
consistently improve MAPE, indicating that simpler seasonal Fourier
terms were sufficient.

\begin{Shaded}
\begin{Highlighting}[]
\CommentTok{\# NNAR + Fourier Model}
\CommentTok{\# I chose to use a Neural Network Autoregressive (NNAR) model with Fourier terms (K = c(2, 12))}
\CommentTok{\# to capture complex seasonal and nonlinear patterns in the daily electricity data.}
\CommentTok{\# I initially experimented with smaller K values, but the fit was too rigid.}
\CommentTok{\# Increasing K allowed the model to flexibly capture both short{-} and long{-}term seasonality.}

\NormalTok{horizon }\OtherTok{\textless{}{-}} \FunctionTok{length}\NormalTok{(ts\_daily\_test)}

\CommentTok{\# Fit the NNAR model with Fourier terms}
\NormalTok{NNAR\_Fourier\_fit }\OtherTok{\textless{}{-}} \FunctionTok{nnetar}\NormalTok{(}
\NormalTok{  ts\_daily\_train, }
  \AttributeTok{p =} \DecValTok{2}\NormalTok{, }\AttributeTok{P =} \DecValTok{2}\NormalTok{, }
  \AttributeTok{xreg =} \FunctionTok{fourier}\NormalTok{(ts\_daily\_train, }\AttributeTok{K =} \FunctionTok{c}\NormalTok{(}\DecValTok{2}\NormalTok{, }\DecValTok{12}\NormalTok{))}
\NormalTok{)}

\CommentTok{\# Forecast using the fitted model}
\NormalTok{NNAR\_Fourier\_forecast }\OtherTok{\textless{}{-}} \FunctionTok{forecast}\NormalTok{(}
\NormalTok{  NNAR\_Fourier\_fit, }
  \AttributeTok{h =}\NormalTok{ horizon, }
  \AttributeTok{xreg =} \FunctionTok{fourier}\NormalTok{(ts\_daily\_train, }\AttributeTok{K =} \FunctionTok{c}\NormalTok{(}\DecValTok{2}\NormalTok{, }\DecValTok{12}\NormalTok{), }\AttributeTok{h =}\NormalTok{ horizon)}
\NormalTok{)}

\CommentTok{\# Plot the forecast against the test data}
\CommentTok{\# This visual check helped me confirm that the forecast captured both the trend and seasonal fluctuation well.}
\FunctionTok{autoplot}\NormalTok{(ts\_daily\_test) }\SpecialCharTok{+} 
  \FunctionTok{autolayer}\NormalTok{(NNAR\_Fourier\_forecast, }\AttributeTok{series =} \StringTok{"NNAR + Fourier (K=2,12)"}\NormalTok{)}
\end{Highlighting}
\end{Shaded}

\begin{center}\includegraphics{Bhuyan_Chuang_Martinez_CompetitionReport_S25_files/figure-latex/nnar_fourier_k212-1} \end{center}

\begin{Shaded}
\begin{Highlighting}[]
\CommentTok{\# Calculate accuracy metrics}

\FunctionTok{accuracy}\NormalTok{(NNAR\_Fourier\_forecast, ts\_daily\_test) }\CommentTok{\# Score 23.48}
\end{Highlighting}
\end{Shaded}

\begin{verbatim}
##                     ME      RMSE       MAE        MPE      MAPE      MASE
## Training set  -1.19979  154.2441  112.2034 -0.4953418  3.408601 0.1458308
## Test set     733.65227 2055.7626 1731.3576  6.7714327 34.432443 2.2502467
##                    ACF1 Theil's U
## Training set -0.1041646        NA
## Test set      0.8504793  2.478301
\end{verbatim}

\subsubsection{Model 4: NNAR + Fourier (K =
3,18)}\label{model-4-nnar-fourier-k-318}

We tested NNAR + Fourier (K = 3,18) to explore whether adding more
seasonal components would improve forecast accuracy. By increasing the
number of Fourier terms, the model could capture more detailed seasonal
fluctuations and smaller periodic patterns that might not be fully
represented with lower K values.

\begin{Shaded}
\begin{Highlighting}[]
\NormalTok{K4 }\OtherTok{\textless{}{-}} \FunctionTok{c}\NormalTok{(}\DecValTok{3}\NormalTok{,}\DecValTok{18}\NormalTok{)}
\NormalTok{NN\_fit\_k318 }\OtherTok{\textless{}{-}} \FunctionTok{nnetar}\NormalTok{(ts\_daily\_train, }\AttributeTok{p =} \DecValTok{2}\NormalTok{, }\AttributeTok{P =} \DecValTok{2}\NormalTok{, }\AttributeTok{xreg =} \FunctionTok{fourier}\NormalTok{(ts\_daily\_train, }\AttributeTok{K =}\NormalTok{ K4), }\AttributeTok{size =} \DecValTok{10}\NormalTok{, }\AttributeTok{decay =} \FloatTok{0.01}\NormalTok{, }\AttributeTok{maxNWts =} \DecValTok{2000}\NormalTok{)}
\NormalTok{NN\_for\_k318 }\OtherTok{\textless{}{-}} \FunctionTok{forecast}\NormalTok{(NN\_fit\_k318, }\AttributeTok{h =}\NormalTok{ horizon, }\AttributeTok{xreg =} \FunctionTok{fourier}\NormalTok{(ts\_daily\_train, }\AttributeTok{K =}\NormalTok{ K4, }\AttributeTok{h =}\NormalTok{ horizon))}
\FunctionTok{autoplot}\NormalTok{(ts\_daily\_test) }\SpecialCharTok{+} \FunctionTok{autolayer}\NormalTok{(NN\_for\_k318, }\AttributeTok{series =} \StringTok{"Model 4 Forecast"}\NormalTok{)}
\end{Highlighting}
\end{Shaded}

\begin{center}\includegraphics{Bhuyan_Chuang_Martinez_CompetitionReport_S25_files/figure-latex/model4-nnar-k318-1} \end{center}

\begin{Shaded}
\begin{Highlighting}[]
\FunctionTok{accuracy}\NormalTok{(NN\_for\_k318, ts\_daily\_test)}
\end{Highlighting}
\end{Shaded}

\begin{verbatim}
##                        ME      RMSE      MAE        MPE      MAPE     MASE
## Training set  -0.04863108  216.0197  152.232 -0.6791729  4.499708 0.197856
## Test set     832.16963282 1903.3639 1639.336  9.3756694 32.468027 2.130646
##                     ACF1 Theil's U
## Training set -0.06335298        NA
## Test set      0.84792922  2.306262
\end{verbatim}

\subsubsection{Model 5: TBATS}\label{model-5-tbats}

We tested TBATS because it is designed to handle seasonal patterns,
multiple seasonalities, and nonstationary behavior. In our case, TBATS
did not perform as well as the NNAR + Fourier models in the Kaggle
competition as load was just better forecasted by a more targeted
combination of Fourier-seasonality and a nonlinear neural network,
whereas TBATS oversmoothed or lagged behind sharp changes in load.

\begin{Shaded}
\begin{Highlighting}[]
\NormalTok{TBATS\_fit }\OtherTok{\textless{}{-}} \FunctionTok{tbats}\NormalTok{(ts\_daily\_train)}
\NormalTok{TBATS\_for }\OtherTok{\textless{}{-}} \FunctionTok{forecast}\NormalTok{(TBATS\_fit, }\AttributeTok{h =}\NormalTok{ horizon)}
\FunctionTok{autoplot}\NormalTok{(ts\_daily\_test) }\SpecialCharTok{+} \FunctionTok{autolayer}\NormalTok{(TBATS\_for, }\AttributeTok{series =} \StringTok{"Model 5 Forecast"}\NormalTok{)}
\end{Highlighting}
\end{Shaded}

\begin{center}\includegraphics{Bhuyan_Chuang_Martinez_CompetitionReport_S25_files/figure-latex/model5-tbats-1} \end{center}

\begin{Shaded}
\begin{Highlighting}[]
\FunctionTok{accuracy}\NormalTok{(TBATS\_for, ts\_daily\_test)}
\end{Highlighting}
\end{Shaded}

\begin{verbatim}
##                     ME      RMSE       MAE        MPE     MAPE      MASE
## Training set  43.90081  501.3385  363.7827 -0.6194292 10.78758 0.4728086
## Test set     711.90657 1376.0978 1083.3750  9.2346813 20.45227 1.4080633
##                    ACF1 Theil's U
## Training set 0.03361668        NA
## Test set     0.79537200  1.553679
\end{verbatim}

\subsection{Model Comparison Table}\label{model-comparison-table}

\subsubsection{Model Evaluation
Discussion}\label{model-evaluation-discussion}

Based on the validation MAPE values, \textbf{Model 1: NNAR + Fourier
(K=2,8)} achieved the best performance. It effectively captured weekly
and annual seasonal patterns while maintaining model simplicity.

\begin{itemize}
\tightlist
\item
  \textbf{Model 1 (K=2,8)} showed strong generalization by balancing
  underfitting and overfitting, and leveraging just enough Fourier terms
  to capture dominant seasonality.
\item
  \textbf{Model 2 (K=2,12)} slightly overfitted to noise, leading to a
  marginal increase in MAPE despite capturing more complex patterns.
\item
  \textbf{Model 3 (K=2,12) + Temp} demonstrated that including external
  regressors such as temperature can introduce multicollinearity,
  complicating the model without consistent forecasting gains.
\item
  \textbf{Model 4 (K=3,18)} increased the model's flexibility but also
  risked overfitting due to high model complexity without significant
  improvement in predictive accuracy.
\item
  \textbf{Model 5 (TBATS)} handled seasonality flexibly but did not
  outperform simpler NNAR + Fourier models, indicating the electricity
  demand series had relatively stable seasonality well captured by
  Fourier harmonics.
\end{itemize}

Overall, simpler seasonal structure combined with neural networks
provided the best generalization to unseen data.

\begin{longtable}[]{@{}
  >{\raggedright\arraybackslash}p{(\columnwidth - 6\tabcolsep) * \real{0.6528}}
  >{\raggedleft\arraybackslash}p{(\columnwidth - 6\tabcolsep) * \real{0.1250}}
  >{\raggedleft\arraybackslash}p{(\columnwidth - 6\tabcolsep) * \real{0.1250}}
  >{\raggedleft\arraybackslash}p{(\columnwidth - 6\tabcolsep) * \real{0.0972}}@{}}
\caption{Performance Comparison of All 5 Models (Train/Test
Evaluation)}\tabularnewline
\toprule\noalign{}
\begin{minipage}[b]{\linewidth}\raggedright
Model
\end{minipage} & \begin{minipage}[b]{\linewidth}\raggedleft
RMSE
\end{minipage} & \begin{minipage}[b]{\linewidth}\raggedleft
MAE
\end{minipage} & \begin{minipage}[b]{\linewidth}\raggedleft
MAPE
\end{minipage} \\
\midrule\noalign{}
\endfirsthead
\toprule\noalign{}
\begin{minipage}[b]{\linewidth}\raggedright
Model
\end{minipage} & \begin{minipage}[b]{\linewidth}\raggedleft
RMSE
\end{minipage} & \begin{minipage}[b]{\linewidth}\raggedleft
MAE
\end{minipage} & \begin{minipage}[b]{\linewidth}\raggedleft
MAPE
\end{minipage} \\
\midrule\noalign{}
\endhead
\bottomrule\noalign{}
\endlastfoot
Model 1: NNAR + Fourier (K = c(2,8)) & 1967.805 & 1642.677 & 32.499 \\
Model 2: NNAR + Fourier (K = c(2,12)) Baseline & 2038.306 & 1737.296 &
34.338 \\
Model 3: NNAR + Fourier (K = c(2,12)) on Test & 2055.763 & 1731.358 &
34.432 \\
Model 4: NNAR + Fourier (K = c(3,18)) & 1903.364 & 1639.336 & 32.468 \\
Model 5: TBATS & 1376.098 & 1083.375 & 20.452 \\
\end{longtable}

\subsection{Final Forecast for 2011}\label{final-forecast-for-2011}

\begin{Shaded}
\begin{Highlighting}[]
\CommentTok{\# Define full time series}
\NormalTok{ts\_full }\OtherTok{\textless{}{-}}\NormalTok{ ts\_electricity\_daily}
\NormalTok{final\_dates }\OtherTok{\textless{}{-}} \FunctionTok{seq}\NormalTok{(}\FunctionTok{as.Date}\NormalTok{(}\StringTok{"2011{-}01{-}01"}\NormalTok{), }\FunctionTok{as.Date}\NormalTok{(}\StringTok{"2011{-}02{-}28"}\NormalTok{), }\AttributeTok{by =} \StringTok{"day"}\NormalTok{)}
\NormalTok{final\_horizon }\OtherTok{\textless{}{-}} \FunctionTok{length}\NormalTok{(final\_dates)}

\CommentTok{\# Create Fourier regressors}
\NormalTok{K1 }\OtherTok{\textless{}{-}} \FunctionTok{c}\NormalTok{(}\DecValTok{2}\NormalTok{,}\DecValTok{8}\NormalTok{)}
\NormalTok{xreg\_full\_1 }\OtherTok{\textless{}{-}} \FunctionTok{fourier}\NormalTok{(ts\_full, }\AttributeTok{K =}\NormalTok{ K1)}
\NormalTok{xreg\_fc\_1 }\OtherTok{\textless{}{-}} \FunctionTok{fourier}\NormalTok{(ts\_full, }\AttributeTok{K =}\NormalTok{ K1, }\AttributeTok{h =}\NormalTok{ final\_horizon)}

\CommentTok{\# Fit Model 1 on full data}
\NormalTok{fit\_nnar\_full }\OtherTok{\textless{}{-}} \FunctionTok{nnetar}\NormalTok{(ts\_full, }\AttributeTok{p =} \DecValTok{2}\NormalTok{, }\AttributeTok{P =} \DecValTok{2}\NormalTok{, }\AttributeTok{xreg =}\NormalTok{ xreg\_full\_1)}

\CommentTok{\# Forecast for Jan{-}Feb 2011}
\NormalTok{fc\_nnar\_full }\OtherTok{\textless{}{-}} \FunctionTok{forecast}\NormalTok{(fit\_nnar\_full, }\AttributeTok{h =}\NormalTok{ final\_horizon, }\AttributeTok{xreg =}\NormalTok{ xreg\_fc\_1)}

\CommentTok{\# Save forecast}
\NormalTok{final\_forecast\_df }\OtherTok{\textless{}{-}} \FunctionTok{data.frame}\NormalTok{(}\AttributeTok{date =}\NormalTok{ final\_dates, }\AttributeTok{load =} \FunctionTok{as.numeric}\NormalTok{(fc\_nnar\_full}\SpecialCharTok{$}\NormalTok{mean))}
\FunctionTok{autoplot}\NormalTok{(fc\_nnar\_full) }\SpecialCharTok{+} \FunctionTok{ggtitle}\NormalTok{(}\StringTok{"Final Forecast Using Model 1: NNAR + Fourier (K=2,8)"}\NormalTok{)}
\end{Highlighting}
\end{Shaded}

\begin{center}\includegraphics{Bhuyan_Chuang_Martinez_CompetitionReport_S25_files/figure-latex/final-forecast-model1-1} \end{center}

The final model selection was \textbf{Model 1: NNAR + Fourier (K=2,8)}
based on lowest validation MAPE when forecasting 2011. We retrained this
model using the full dataset (2005--2010) and forecasted daily load for
January 1 to February 28, 2011.

\subsection{Conclusion}\label{conclusion}

Through systematic model development and evaluation, we determined that
\textbf{Model 1: NNAR + Fourier (K=2,8)} provided the most accurate
forecasts. It effectively captured both weekly and yearly seasonal
patterns while maintaining simplicity. Models with more Fourier terms or
additional temperature regressors did not outperform this baseline.

Our Kaggle submissions demonstrated steady improvement, culminating in a
final forecast that surpassed the vanilla STL+ETS benchmark.

\subsection{Acknowledgment of AI
Assistance}\label{acknowledgment-of-ai-assistance}

ChatGPT was used to assist with R Markdown formatting, report
organization, and code cleaning. All modeling decisions, model
selection, and data handling were conducted independently by the project
team.

\end{document}
